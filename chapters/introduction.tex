\chapter{Johdanto}

Kuvitellaan, että aloitteleva kehittäjä tekee ensimmäistä tietokannallista nettisivuprojektiaan. Koska kyseessä on ensimmäinen tietokannallinen nettisivu niin sitä kehitetään PHP:lla ilman ulkoisia kirjastoja. PHP-koodi ajetaan lokaalilla Apachella ja kaikki sujuu yleisesti hyvin. Projektiin liittyy myöhemmin toinenkin kehittäjä. Uusi kehittäjä pääsee helposti mukaan, sillä kehitysympäristö on vielä yksinkertainen, koska projektissa käytetään perus LAMP-ympäristöä ja ulkoisia kirjastoja ei ole.

Myöhemmin kun aloittava kehittäjä tekee uutta Node-projektia ja kutsuu uuden kehittäjän mukaan tiimiin. Vanhasta kokemuksesta kehitysympäristön pitäisi toimia yhtä hyvin kuin PHP-projektissakin. Kehittäjä opastaa tulokkaalle, että asentaa Node:n, kloona projektin ja asentaa riippuvuudet. Tämän tehtyää uusi kehittäjä huomaa että riippuvuudet eivät asennu. Molemmat kehittäjävät selvittelivät vikaa muutaman päivän ajan ja lopulta selvisi, että uudella kehittäjällä oli uudempi versio Nodesta, jota yksi riippuvuuksista ei tukenut.

Tämä inssityö auttaa ratkaisemaan tämänkaltaisia ongelmia kehitysympäristön automatisoinnilla. Insinöörityössä kerrotaan mitä ovat Vagrant ja Ansible ja miten niillä voidaan ratkaista yleisimpiä kehitysympäristö ja tuotantoon viemisen ongelmia.
