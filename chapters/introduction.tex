\chapter{Johdanto}

Kuvitellaan, että aloitteleva kehittäjä Klaus tekee ensimmäistä tietokannallista nettisivuprojektia PHP:lla ilman ulkoisia kirjastoja. PHP-koodi ajetaan paikallisella Apache-palvelimella ja kaikki sujuu ongelmitta. Projektiin liittyy myöhemmin toinen kehittäjä Pekka. Uusi kehittäjä pääsee helposti mukaan, sillä kehitysympäristö on vielä yksinkertainen. Projektista yksinkertaisen tekee yleisesti käytetty käytetty LAMP-kehitysympäristö ja se, että ulkoisia kirjastoja ei toistaiseksi ole.

Myöhemmin Klaus aloittaa uuden Node-projektin ja kutsuu Pekan jälleen mukaan tiimiin. Vanhasta kokemuksesta kehitysympäristön pitäisi toimia yhtä hyvin kuin edellisessä PHP-projektissakin. Klaus antaa Pekalle seuraavat ohjeet: asenna Node, kloonaa projekti ja asenna riippuvuudet. Tämän tehtyää Pekka huomaa, että riippuvuudet eivät asennu. Eikä ongelman ratkaisemiseksi auta kryptinen virheilmoituskaan. Molemmat kehittäjät selvittävät vikaa muutaman päivän ajan ja lopulta selvisi, että Pekalla oli uudempi versio Nodesta, jota yksi riippuvuuksista ei tukenut.

Tämä inssityö auttaa ratkaisemaan samankaltaisia ongelmia tekemällä  kehitysympäristön asentamisesta automaattisen. Tämä käytännössä tarkoittaa, että kehitysympäristö on virtuaalinen, joten omalle fyysiselle tietokoneelle asenneta mitään projektikohtaisia riippuvuuksia. Insinöörityössä kerrotaan mitä ovat Vagrant ja Ansible ja miten niillä voidaan ratkaista yleisimpiä kehitysympäristö ja tuotantoon viemisen ongelmia.

Insinöörityön tavoitteena on luoda virtuaalisiakehitysympäristöjä, joita voidaan jakaa projektin jäsenien kesken ja joita helppo pystyttää ilman oman koneen saastuttamista.
