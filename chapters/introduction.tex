\chapter{Johdanto}

Kuvitellaan, että aloitteleva kehittäjä Klaus tekee ensimmäistä tietokannallista nettisivuprojektia PHP:lla ilman ulkoisia kirjastoja. PHP-koodi ajetaan paikallisella Apachella ja kaikki sujuu yleisesti hyvin. Projektiin liittyy myöhemmin toinenkin kehittäjä Pekka. Uusi kehittäjä pääsee helposti mukaan, sillä kehitysympäristö on vielä yksinkertainen. Projektista tekee yksinkertaisen yleisesti käytetty käytetty LAMP-kehitysympäristö ja ulkoisten kirjastojen puuttuminen.

Myöhemmin Klaus aloittaa uuden Node-projektin ja kutsuu Pekan jälleen mukaan tiimiin. Vanhasta kokemuksesta kehitysympäristön pitäisi toimia yhtä hyvin kuin edellisessä PHP-projektissakin. Klaus antaa pekalle seuraavat ohjeet: asenna Node, kloonaa projekti ja asenna riippuvuudet. Tämän tehtyää Pekka huomaa, että riippuvuudet eivät asennu. Eikä ongelman ratkaisemiseksi auta kryptinen virheilmoituskaan. Molemmat kehittäjävät selvittelivät vikaa muutaman päivän ajan ja lopulta selvisi, että Pekalla oli uudempi versio Nodesta, jota yksi riippuvuuksista ei tukenut.

Tämä inssityö auttaa ratkaisemaan samankaltaisia ongelmia tekemällä  kehitysympäristön asentamisesta automaattisen. Insinöörityössä kerrotaan mitä ovat Vagrant ja Ansible ja miten niillä voidaan ratkaista yleisimpiä kehitysympäristö ja tuotantoon viemisen ongelmia.
