\chapter{Yhteenveto}

Insinöörityön tarkoituksena oli ratkaista yleisimpiä kehitysympäristöön liittyviä ongelmia. Suurin osa näistä ongelmista johtuu paikalliseen koneeseen asennetuista käyttöjärjestelmäherkistä ohjelmista. Näin syntyy usein \enquote{toimii minun koneellani -tilanteita}. Insinöörityössä esitettiin ratkaisuja miten erilaisia kehitysympäristöjä virtualisoimalla päästään eroon yleisimmistä ongelmista.

Työssä käytettiin Vagrant-ohjelmaa luomaan yhtenäisiä virtuaalikoneita, joiden luomisprosessi voidaan kuvata koodina ja jotka voidaan versioida projektin versionhallintaan. Näin virtuaalikoneet ovat helposti jaettavia tiimin kesken. Ensiksi kerrottiin vertauskuvan ja kaavioiden avulla miten Vagrant toimii. Sen jälkeen käytiin yksityiskohtaisesti käytännön tasolla läpi miten voidaan luoda yksinkertainen virtuaalikone Apache- ja Node-ympäristöön. Insinöörityössä mainittiin myös Vagrantin huonot puolet kuten raskas muistin käyttö ja ongelmat isäntäkoneen ollessa Windows.

Insinöörityön jälkimmäisessä osassa tutustuttiin Ansibleen, joka helpottaa virtuaalikoneiden asennusta verrattuna shell-skripteihin. Ansiblen tapauksessakin ensi selitettiin Ansiblen hyöty vertauskuvana ja sen jälkeen sama asia selitettiin kaavioiden avulla. Selitysten jälkeen paranneltiin Vagrant osiossa luotuja virtuaalikoneita muuttamalla sen ohjelmien asennus Ansiblen mukaisiksi. Lopputuloksena oli  hyvin modulisoitu playbook, jonka osia voi helposti käyttää muissakin projekteissa, joissa tarvitaan samoja ohjelmia kuin testiprojektissa.

Kaiken kaikkiaan insinöörityö on antanut lukijalle valmiudet jatkaa virtuaaliympäristöjen tutkimista. Virtuaaliympäristöjen hyödyt kannattaa ottaa heti käyttöön, sillä nykyisillä työmarkkinoilla näiden taitojen hallitseminen on tärkeää.