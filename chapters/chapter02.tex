\chapter{Ansible}

Ansible on tarkoitettu palvelinten konfiguraation hallinnoimiseen, palveluiden hallinnointiin ja esimerkiksi web-palveluiden käyttöönottoon ja asentamiseen. Ansiblella voidaan esimerkiksi asentaa kehitys- että tuotantoympäristöön tarvittavat riippuvuudet helposti ja paremmin kuin esimerkiksi shell-skripteillä. Nykyään on myös hyvä tapa tehdä kehitys- että tuotantonympäristöstä mahdollisimman samanlaisia, jotta ympäristöjen erilaisuuksista ei tule ongelmia. Ansiblessa pystyy helposti uudelleen käyttämään siihen määriteltyjä asennustehtäviä parametrisoimalla niitä. Esimerkiksi tietokantayhteys voi hieman poiketa tuotantoympäristössä. Kaikki nämä ja paljon muuta hoituu helposti Ansiblella.

Mietitään jälleen vertauskuvaa Subway-ravintolan perustamisesta. Kun laatikosta tullut robotti on rakentanut ravintolan liiketilan se aloittaa laitteiden asentamisen ja sisustamisen. Jos Erik haluaa myöhemmin esimerkiksi sisustaa ravintolan uudestaan ohjeiden mukaan ollakseen varma, että ravintolan on Subwayn johdon määritelmän mukainen, hänen täytyy kutsua robottia tekemään sisustus uudestaan. Tämä toiminpide täytyy tehdä viimeistään silloin kun Subwayn johto on muuttanut rakennusohjeita.

Vanha robotti tekisi tällöin sekä laitteiden asennuksen, että sisustuksen uudestaan, mutta se saattaa rikkoa paikkoja, koska se esimerkiksi sisustaisi kaiken uudestaan vaikka osa sisustuksesta olisikin jo paikoillaan. Tästä voi luonnollisesti koitua ongelmia ja pahimmassa tapauksessa robotin täytyy tuhota koko ravintola ja aloittaa alusta rakentaminen, laitteiden asennus ja sisustaminen. Toisaalta tämä ei haittaa, koska alusta asti rakennettu ravintola toimii aina. Ainostaan aikaa menee tällöin hukkaan.

Uusi robotti on fiksumpi kuin edellinen. Kun Erik haluaa varmistaa, että ravintola on vanhojen ohjeiden mukainen tai varmistaakseen, että uusimmat muutokset ovat tehty ravintolaan täytyy hänen jälleen kutsua robottia asentamaan ravintola. Ennen kuin uusi robotti alkaa asentaa mitään se tutkii ravintolaa ja kerää siitä tietoa. Kun tiedot on kerätty se aloittaa asentamisen. Esimerkiksi jääkaapin asennuken kohdalla robotti ensin tutkii onko jääkaappi jo asennettu ja jos se on niin se siirtyy seuraavaan asennukseen. Asennuksen päätteeksi robotti ilmoittaa Erikille kuinka monta asiaa asennuslistalta oli jo tehty ja kuinka monta asiaa muutettiin. Esimerkiksi: "10 kohdetta oli jo asennettu ja jääkaapin tilalle asennettiin jääkaappipakastin yhdistelmä".

\section{Ansiblen toiminta}

Ansiblen kaltaisia ohjelmia on muutamia, mutta Ansiblen suosio perustuu sen yksinkertaisuuteen ja keveyteen. Käyttääksesi Ansiblea tarvitset vain Pythonin, SSH-yhteyden kohdekoneeseen ja käyttäjän, joka voi suorittaa skriptejä kohdekoneella \cite{link:what-is-ansible}. Jotta Ansiblen käyttäminen olisi erittäin sujuvaa kannattaa kohdekoneelle lisätä SSH-avain, koska tällöin ei tarvita käyttäjätunnus salasana yhdistelmää kun otetaan yhteyttä kohdekoneeseen.

Jotta Ansible pääsee suorittamaan sille annettuja tehtäviä kohdekoneeseen, täytyy sen ensin ottaa SSH-yhteys kohdekoneeseen. Tämä ei vaadi client-koneelta yleisesti mitään lisäasetuksia. Kun yhteys on saatu kohdekoneeseen kerää Ansible tietoja siitä esimerkiksi käyttäjärjestelmän, mitä paketteja on jo asennettu yms. Sen jälkeen Ansible lähtee ajamaan sille määriteltyjä tehtäviä kohdekoneelle.

\begin{figure}[h]
  \includegraphics[width=\textwidth]{how-ansible-works}
  \caption{Ansiblen toiminta}
  \label{fig:how-ansible-works}
\end{figure}

Tehtävä voi olla esimerkiksi palvelinohjelman kuten Apachen asentaminen, tiedoston kopioiminen tai tietokannan käynnistäminen. Ansible suorittaa sille annetut tehtävät määritellyssä järjestyksessä. Tehtävät kuvataan Ansiblen tarjoamilla moduuleilla. Moduuleita on niin paljon, että varmasti jokaiseen tarpeeseen löytyy oma. Ansiblelta löytyy hyvät dokumentaatiot kaikille moduuleille. Tiedoston kopioiminen onnistuu helposti moduulilla copy-moduulilla. Kun SSH-yhteys on luotu ja kohde konetta on tutkittu, voidaan aloittaa tehtävien suorittaminen.

Kuvassa \ref{fig:how-ansible-loops-tasks} näytetään miten Ansible prosessoi sille annettuja tehtäviä. Ansible tarkistaa kaikkien tehtävien kohdalla on muutoksia tapahtunut. Muutos tarkoittaa, että onko tehtävää suoritettu ollenkaan tai onko tehtävää muutettu sitten viime suorituksen jälkeen. Esimerkiksi jos Apachea ei ole asennettu ollenkaan niin suoritetaan tehtävä. Lisäksi jos viimeksi sama tehtävä on muutettu asentamaan NGINX Apachen sijaan niin suoritetaan tehtävä tällöinkin. Kun tehtävä on suoritettu tarkistetaan onko vielä tehtäviä jäljellä. Jos on niin jatketaan niiden suorittamista. Jos Ansible toteaa, että muutoksia ei ole eli esimerkiksi kyseinen ohjelma on jo asennettu, jatketaan suoraan tarkistamaan onko tehtäviä vielä jäljellä -osioon. Tätä silmukkaa jatketaan kunnes tehtäviä ei ole enään jäljellä.

Kun tehtävä ovat loppuneet antaa Ansible yhteenvedon tehtävistä. Se kertoo esimerkiksi kuinka monta tehtävää ei tarvinnut suorittaa (ei muutoksia) ja kuinka tehtävää suoritettiin (muutoksia oli). Yksi Ansiblen hienoista ominaisuuksista on päivityksen peruminen. Jos äsken suoritettu asennus rikkoi kohdekoneen niin päivityksen voi vielä peruuttaa. Peruuttamisen avulla kehittäjät voivat huoletta viedä uusia ominaisuuksia tuotantoon ilman pelkoa, että kaikki menee hajalle. Aikaisemmin tällaisesta tilanteesta olisi voinut koitua erittäin suurta vaivaa saada kohdekoneen tila samaa kuin se oli aikaisemmin.

\begin{figure}[h]
  \centering
  \includegraphics[width=\textwidth, height=\textheight,keepaspectratio]{how-ansible-loops-tasks}
  \caption{Ansiblen tehtäväprosessointi}
  \label{fig:how-ansible-loops-tasks}
\end{figure}

\section{Ansiblen käyttäminen}

Ansiblen käytön aloittamiseksi se pitää asentaa osoiteen \url{http://docs.ansible.com/ansible/intro_installation.html} mukaan. Asennuksen jälkeen avaa komentorivi ja komenna \code{ansible --version} ja jos Ansible kertoo nykyisen versionsa on Ansible asennettu oikein.

Seuraavaksi luodaan jälleen Vagrant virtuaalikone, mutta nyt siihen asennettavat ohjelmat ja riippuvuudet ei asenneta shell-skriptin vaan Ansible playbookin avulla. Jatketaan kuviossa \ref{listing:vagrant-final-apache-setup} olevaa Vagrantfileä.

Ensiksi korvataan shell-skriptin määrittelevä rivi, niin että määrittelemme riippuvuuksien asentajaksi Ansible. Kuviosta \ref{listing:vagrantfile-ansible} nähdään Ansible määrittely. Määrittelyssä kerrotaan Ansiblelle, että playbook löytyy tiedostosta "playbook.yml", joka on samassa kansiossa, jossa Vagrantfile on. Playbook sisältää kaikki tiedot virtuaalikoneen riippuvuuksien asentamiseen. Playbook-tiedosto on yksinkertainen YAML-tiedosto.

\begin{lstlisting}[
  label=listing:vagrantfile-ansible,
  language=Ruby,
  caption=Määritellään virtuaalikoneen riippuvuuksien asentajaksi Ansible,
  float=h
]
config.vm.provision "ansible" do |ansible|
  ansible.playbook = "playbook.yml"
end
\end{lstlisting}

Sitten luodaan playbook.yml tiedosto, joka on kuvattu kuviossa \ref{listing:apache-playbook}. Kaikki YAML-tiedostot pitäisi aloittaa sijoittamalla kolme väliviivaa dokumentin alkuun. Tämä on YAML:n tapa kertoa, dokumentin alkukohdasta.

Ensin määritellään hostit eli mille kohdekoneille kyseinen playbook suoritetaan. Tässä tapauksessa voidaan määritellä kaikki kohdekoneet, koska Ansiblea käskytetään Vagrantin kautta, jolloin vain virtuaalikoneelle asennetaan riippuvuuksia \cite{link:comprehensive-ansible-tutorial}. Yleisesti kohteiksi ei laiteta kaikkia mahdollisia kohdekoneita. Kohteina normaalisti olisi esimerkiksi, dev (kehitysympäristö), staging (testausympäristö) tai production (tuotantoympäristö).

Määrittely true avaimelle "become" tarkoittaa, että kaikki playbookissa olevat tehtävät ajetaan root-oikeuksilla. Eli sama asia kuin lisäisi esimerkiksi asennuskomentojen eteen "sudo". Tällä tavalla ohjelmia voidaan ylipäätään asentaa kohdekoneille \cite{link:ansible-configuration-file}.

Jotta voidaan määritellä käyttäjätunnus, jolla tehtäviä ajetaan kohdekoneella täytyy määritelmälle "remote\_user" antaa arvo. Virtuaalikoneen tapauksessa arvoksi laitetaan vagrant, koska se on Vagrant virtuaalikoneiden oletuskäyttäjä.

\begin{lstlisting}[
  label=listing:apache-playbook,
  language=Ruby,
  caption=Pohja Ansible playbookille,
  float=h
]
---
- hosts: all
  become: true
  remote_user: vagrant
  tasks:
    - name: Update apt cache
      apt:
        update_cache: yes
\end{lstlisting}


Muista kertoa ansiblen sivusta galaxy.ansible.com