\fancypagestyle{abstract}{
   \rhead{Tiivistelmä}
   \lfoot{
     \begin{adjustwidth}{-4.2cm}{5cm}
       \includegraphics[]{line-with-metropolia-logo}
     \end{adjustwidth}  
   }
   \setlength{\footskip}{5pt}
}

\chapter*{Abstract}
\thispagestyle{abstract}

\begin{tabular}{|l|l|l|}
  \hline
  Tekijä(t) & Henrik Raitasola \\
  Otsikko & Kehitysympäristön virtualisointi \\
  ~ & ~ \\
  Sivumäärä & \pageref{LastPage} sivua \\
  Aika & \today \\
  \hline
  Tutkinto & Insinööri (AMK) \\
  \hline
  Koulutusohjelma & Tietotekniikka \\
  \hline
  Suuntautumisvaihtoehto & Ohjelmistotekniikka \\
  \hline
  Ohjaaja(t) & Lehtori Ilpo Kuivanen \\
  \hline
  \multicolumn{2}{| p{\textwidth} |}{Tämän insinöörityön aiheena oli ratkaista yleisimpiä kehittäjien ongelmia, jotka liittyvät kehitysympäristöön. Usein ongelmat johtuvat kehittäjien tietokoneiden erilaisuuksista, kuten käyttöjärjestelmästä ja siihen asennettujen projektikohtaisten riippuvuuksien eriävistä versioista. Yleisin lausahdus on "Toimii minun koneellani", kun mystisestä syystä samalla tavalla asennettu kehitysympäristö toimii toisella kehittäjällä ja toisella ei. Ongelmia halutaan ratkaista virtualisoimalla kehitysympäristö. \newline

Kehitysympäristö virtualisoitiin ensin Vagrantin avulla. Vagrant loi virtuaalikoneita, johon asennettiin shell-skriptin avulla halutut projektin riippuvuudet. Sen jälkeen tehtiin monimutkaisempi esimerkki Ansiblen avulla, joka toimii hyvin yhteistyössä Vagrantin kanssa. Ansiblella tehtiin playbookkeja, joilla pystyy selkeämmin jakamaan riippuvuuksien asennus omiin rooleihin. Tämä on selkeämpi lähestymistapa pelkkiin shell-skripteihin verrattuna. \newline

Lopputuloksena esiteltiin muutamia eri kehitysympäristöjen asennuksia. Nämä virtuaaliset kehitysympäristöt minimoivat ongelmia, jotka esiintyvät yleisimmin kun kehittäjillä ei ole yhteistä virtuaalikehitysympäristöä. 

} \\
  \hline
  Avainsanat & Virtualisointi, Vagrant, Ansible \\
  \hline
\end{tabular}