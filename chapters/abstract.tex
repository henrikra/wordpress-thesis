\fancypagestyle{abstract}{
   \rhead{Tiivistelmä}
   \lfoot{
     \begin{adjustwidth}{-4.2cm}{5cm}
       \includegraphics[]{line-with-metropolia-logo}
     \end{adjustwidth}  
   }
   \setlength{\footskip}{5pt}
}

\fancypagestyle{abstract-english}{
   \rhead{Abstract}
   \lfoot{
     \begin{adjustwidth}{-4.2cm}{5cm}
       \includegraphics[]{line-with-metropolia-logo}
     \end{adjustwidth}  
   }
   \setlength{\footskip}{5pt}
}

\chapter*{Tiivistelmä}
\thispagestyle{abstract}

\begin{tabular}{|l|l|l|}
  \hline
  Tekijä(t) & Henrik Raitasola \\
  Otsikko & Kehitysympäristön virtualisointi \\
  ~ & ~ \\
  Sivumäärä & \pageref{LastPage} sivua \\
  Aika & \today \\
  \hline
  Tutkinto & Insinööri (AMK) \\
  \hline
  Koulutusohjelma & Tietotekniikka \\
  \hline
  Suuntautumisvaihtoehto & Ohjelmistotekniikka \\
  \hline
  Ohjaaja(t) & Lehtori Ilpo Kuivanen \\
  \hline
  \multicolumn{2}{| p{\textwidth} |}{Tämän insinöörityön aiheena oli ratkaista yleisimpiä kehittäjien ongelmia, jotka liittyvät kehitysympäristöön. Usein ongelmat johtuvat kehittäjien tietokoneiden erilaisuuksista, kuten käyttöjärjestelmästä ja siihen asennettujen projektikohtaisten riippuvuuksien eriävistä versioista. Yleisin lausahdus on \enquote{Toimii minun koneellani}, kun mystisestä syystä samalla tavalla asennettu kehitysympäristö toimii toisella kehittäjällä ja toisella ei. Ongelmia halutaan ratkaista virtualisoimalla kehitysympäristö. \newline

Kehitysympäristö virtualisoitiin ensin Vagrantin avulla. Vagrant loi virtuaalikoneita, johon asennettiin shell-skriptin avulla halutut projektin riippuvuudet. Sen jälkeen tehtiin monimutkaisempi esimerkki Ansiblen avulla, joka toimii hyvin yhteistyössä Vagrantin kanssa. Ansiblella tehtiin playbookeja, joilla pystyy selkeämmin jakamaan riippuvuuksien asennuksen omiin rooleihin. Tämä on selkeämpi lähestymistapa pelkkiin shell-skripteihin verrattuna. \newline

Lopputuloksena esiteltiin muutamia eri kehitysympäristöjen asennuksia. Nämä virtuaaliset kehitysympäristöt minimoivat ongelmia, jotka esiintyvät yleisimmin, kun kehittäjillä ei ole yhteistä virtuaalikehitysympäristöä. 

} \\
  \hline
  Avainsanat & Virtualisointi, Vagrant, Ansible \\
  \hline
\end{tabular}

\chapter*{Abstract}
\thispagestyle{abstract-english}

\begin{tabular}{|l|l|l|}
  \hline
  Author(s) & Henrik Raitasola \\
  Title & Making Development Environment Virtual \\
  ~ & ~ \\
  Number of pages & \pageref{LastPage} pages \\
  Date & \today \\
  \hline
  Degree & Bachelor of  Engineering \\
  \hline
  Degree Programme & Information Technology \\
  \hline
  Specialisation & Software Engineering \\
  \hline
  Instructor & Ilpo Kuivanen, Senior Lecturer \\
  \hline
  \multicolumn{2}{| p{\textwidth} |}{The purpose of this thesis was to solve the most common development environment problems developers face. Most of the time the problems are caused by the variation of the developer's computers such as operating system. Some of the problems are caused by specific project dependencies. The most common comment is \enquote{It works on my machine!} when for a mystical reason development environment installed in the same way does work on another developer's computer and on the other's it does not. These problems were to be solved by virtualization of the development environment.
\newline

The development environment was first virtualizated with Vagrant. Vagrant created virtual machines in which project dependencies were installed by shell-script. Next, a more complicated example was made with Ansible which worked well with Vagrant. Playbooks were created with Ansible. Playbooks help separating the installation of dependencies to their own roles. This approach is clearer compared to solely shell-scripts.
\newline

As a result a couple of different installations of development environments were introduced. Virtual development environments minimize problems usually caused when developers do not have a common virtual environment available.

} \\
  \hline
  Keywords & virtual, Vagrant, Ansible \\
  \hline
\end{tabular}